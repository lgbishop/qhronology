\documentclass{article}
\pagenumbering{gobble}

\usepackage[
    centering,
    left=2cm,
    right=2cm,
    top=2cm,
    bottom=2cm,
    %includehead,
    %includefoot,
    a4paper,
    %letterpaper
]{geometry}

\usepackage{xparse}
\ExplSyntaxOn
\DeclareExpandableDocumentCommand{\eval}{m}{\fp_eval:n {#1}}
\ExplSyntaxOff

\providecommand\FPS{50} % Frame rate
\providecommand\Period{3} % Time (seconds) of a single loop of the animation
\providecommand\FramesTotal{\eval{\FPS*\Period}} % Total number of frames per loop
\providecommand\FramesCurrent{0} % Current frame number, defaults to 0

\providecommand\WormholeCircles{45} % Number of rings in the mesh
\providecommand\WormholeSectors{30} % 32 for centered line at zero rotation
\providecommand\WormholeRotations{2} % How many degrees to rotate per frame

\newcommand\WormholeRate{\eval{\WormholeRotations*(360/\WormholeSectors)/\FramesTotal}} % How many degrees to rotate per frame
\newcommand\WormholeAngle{\eval{90 + \FramesCurrent*\WormholeRate}} % Current rotation position, starting at 0

\providecommand\ClockPositionCyclesHorizontal{3}
\providecommand\ClockPositionCyclesVertical{2}
\providecommand\ClockPositionAmplitudeHorizontal{1}
\providecommand\ClockPositionAmplitudeVertical{1}

\newcommand\ClockShiftHorizontal{sin(\ClockPositionCyclesHorizontal*2*pi*\FramesCurrent/\FramesTotal)}
\newcommand\ClockShiftVertical{sin(\ClockPositionCyclesVertical*2*pi*\FramesCurrent/\FramesTotal)}

\usepackage[utf8]{inputenc}
\usepackage{amsmath,amsfonts,amssymb}
\usepackage{mathrsfs}
%\usepackage{clock}
\usepackage[clock]{ifsym}
\usepackage{mathtools}
\usepackage{dcolumn}% Align table columns on decimal point
\usepackage{bm}
%\usepackage[dvipsnames]{xcolor}
\usepackage{graphicx} % Allows for eps images
\usepackage{url}
\usepackage{enumitem}
\usepackage{tensor}
\usepackage{centernot}
\usepackage{dsfont}
\usepackage[mathlines]{lineno}
\usepackage{tikz}
\usetikzlibrary{arrows,decorations.markings}
\usetikzlibrary{shapes}
\usetikzlibrary{quantikz}

\usepackage[sfdefault]{ClearSans}

\usepackage[T1]{fontenc}
\noindent
%\usepackage{sansmathfonts}

\usepackage[skins]{tcolorbox}
\usepackage{tikz}
\usetikzlibrary{3d,calc,positioning}
\usepackage{pgfplots}
\usepgfplotslibrary{
    colorbrewer,
    colormaps,
    patchplots,
}
\pgfplotsset{width=16cm}
\pgfplotsset{compat=newest}
% Jordy blue = 138,185,241
% Pale cornflour blue = 171,205,239
\pgfplotsset{
    colormap={customcolormap}{[1cm]rgb255(0cm)=(24,115,255); rgb255(1cm)=(30,144,255); rgb255(2cm)=(108,182,255); rgb255(4cm)=(164,214,255); rgb255(6cm)=(189,224,255); rgb255(8cm)=(197,227,255); rgb255(10cm)=(210,232,255)},
    %colormap={customcolormap}{[1cm]rgb255(0cm)=(24,115,255); rgb255(1cm)=(30,144,255); rgb255(2cm)=(145,215,255); rgb255(5cm)=(195,227,255); rgb255(7cm)=(210,232,255)},
    %colormap={customcolormap}{rgb255(0cm)=(30,144,255); rgb(1cm)=(1,1,1)},
    %colormap={customcolormap}{[1cm]rgb255(0cm)=(30,144,255); rgb255(6cm)=(210,232,255); rgb255(12cm)=(210,232,255)},
    colormap={customcolormaplight}{rgb255(0cm)=(171,205,239); rgb(1cm)=(1,1,1); rgb(2cm)=(1,1,1)},
    colormap={customcolormapdark}{rgb(0cm)=(0,1,1); rgb(1cm)=(0,0,0.5); rgb(2cm)=(0,0,0.5)},
    colormap={customcolormapfaces}{rgb255(0cm)=(0, 57, 137),rgb255(1cm)=(0, 57, 137)},
    % colormap={customcolormapmesh}{rgb255(0cm)=(255, 255, 255),rgb255(1cm)=(255, 255, 255),rgb255(2cm)=(0, 57, 137),rgb255(3cm)=(0, 57, 137),rgb255(4cm)=(0, 57, 137)},
    colormap={customcolormapmesh}{rgb255(0cm)=(255, 255, 255),rgb255(1cm)=(255, 255, 255),rgb255(2cm)=(0, 37, 105),rgb255(3cm)=(0, 37, 105)},
}
\newcommand\B{2}
\definecolor{meshcolor}{HTML}{FFFFFF}
\definecolor{meshcolordark}{HTML}{7FFFD4}
\definecolor{meshcolorlight}{HTML}{FFFFFF}
\definecolor{fontcolor}{HTML}{FFFFFF}
\definecolor{maincolor}{HTML}{002569}
% \definecolor{maincolor}{HTML}{003989}
%\definecolor{maincolor}{HTML}{224499}
% \definecolor{maincolor}{RGB}{0,57,137}
\definecolor{fontcolordark}{HTML}{FFFFFF}
\definecolor{fontcolorlight}{HTML}{000000}

\pagecolor{maincolor}

\begin{document}

\begin{tikzpicture}[remember picture, overlay]
    \begin{scope}[xshift=7.8825cm - 1.3cm, yshift=-17.75cm, scale=4.5, rotate=-45, xshift=0cm, yshift=0cm]
    \begin{axis}[
    %hide axis,
    anchor=center,
    %axis equal image,
    unit vector ratio=1 1 1,
    plot box ratio=1 1 4,
    view={0}{42},
    line width=12000sp, % 2000 to 7500
    data cs=polar,
    samples=(\WormholeSectors+1),
    samples y=\WormholeCircles, %circles
    domain=\WormholeAngle:(360+\WormholeAngle),
    y domain=\B:500,
    colormap name=customcolormapmesh,
    declare function={
        wormhole(\r)={2*\B*sqrt(\r/\B - 1)};
        % added functions to calculate cartesian coordinates from
        % polar coordinates
        pol2cartX(\angle,\radius) = \radius * cos(\angle);
        pol2cartY(\angle,\radius) = \radius * sin(\angle);
    },
    ]
    %\addplot3 [surf, shader=flat, draw=black, z buffer=sort, samples=50] {-wormhole(y)};
    \IfFileExists{./clock-distorted.png}{ % True

    }{ % False
    \addplot3 [surf, shader=flat, fill=maincolor, fill opacity=1, z buffer=sort] {wormhole(1.5*y)};
    }
    %\addplot3 [surf, shader=interp, z buffer=sort, samples=50] {wormhole(y)};
    %\addplot3 [surf, smooth, draw=black, z buffer=sort, samples=50] {wormhole(y)};
    %\node at (60,wormhole(200)) {\includegraphics[width=12cm]{extra_fancy_clock.png}};
    \end{axis}
    \end{scope}

    % \filldraw [fill=maincolor, draw=maincolor] (-10,10) rectangle (10.5+20,-14.85+4.5);
    % \filldraw [fill=maincolor, draw=maincolor] (-10,10) rectangle (-0.75,-30);
    % \filldraw [fill=maincolor, draw=maincolor] (30,10) rectangle (20 - 3.25,-30);
    % \filldraw [fill=maincolor, draw=maincolor] (-10,-30) rectangle (10.5+20,-24.7+1);

    \IfFileExists{./clock-distorted.png}{ % True
    \node[anchor=center,xshift=13.175cm,yshift=-1.35cm,rotate=45, opacity=1] at (0,0) {\includegraphics[height=1.5cm]{clock-distorted.png}};
    }{ % False
    \node[anchor=center,xshift=\eval{6.4825 + 1.5 + 0.125*\ClockPositionAmplitudeHorizontal*\ClockShiftHorizontal} cm,yshift=\eval{-16.35 - 0.125*\ClockPositionAmplitudeVertical*\ClockShiftVertical} cm,rotate=-45, opacity=1] at (0,0) {\includegraphics[height=6.5cm]{render-clock-fixed.png}};
    % \node[anchor=center,xshift=6.75cm,yshift=-1.00cm,rotate=0, scale=8.5, color=fontcolor] at (0,0) {\fontfamily{qtm}\selectfont \contour{fontcolor}{Q}\tikzunderarrow[fontcolor]{\smash{\contour{fontcolor}{h\hspace{0.01cm}ronolo}}}\contour{fontcolor}{gy}};
    % \node[anchor=north west,xshift=-0.5cm,yshift=0.5cm,rotate=0, scale=0.6, color=fontcolor] at (0,0) {\includegraphics{logo-text.pdf}};
 %    \node[anchor=west,xshift=-0.5cm,yshift=-3.25cm,rotate=0, scale=1.45, color=fontcolor] at (0,0) {\textbf{A Python package for resolving quantum time-travel paradoxes}};
    % \node[anchor=west,xshift=-0.5cm,yshift=-4.00cm,rotate=0, scale=1.45, color=fontcolor] at (0,0) {\textbf{and performing general quantum information processing \& computation}};
 %    \node[anchor=west,xshift=-0.75cm,yshift=-5.75cm,rotate=0, scale=3, color=fontcolor] at (0,0) {\textbf{Documentation,}};
 %    \node[anchor=west,xshift=-0.75cm,yshift=-7.25cm,rotate=0, scale=3, color=fontcolor] at (0,0) {\textbf{Examples,}};
 %    \node[anchor=west,xshift=-0.75cm,yshift=-8.75cm,rotate=0, scale=3, color=fontcolor] at (0,0) {\textbf{and Theory}};
    % \node[anchor=west,xshift=-0.75cm,yshift=-25.15cm,rotate=0, scale=2.5, color=fontcolor] at (10,0) {\textbf{Lachlan G. Bishop}};
    }
\end{tikzpicture}

\end{document}

