\documentclass{article}
\pagenumbering{gobble}

\usepackage{xparse}
\ExplSyntaxOn
\DeclareExpandableDocumentCommand{\eval}{m}{\fp_eval:n {#1}}
\ExplSyntaxOff

\providecommand\FPS{50} % Frame rate
\providecommand\Period{3} % Time (seconds) of a single loop of the animation
\providecommand\FramesTotal{\eval{\FPS*\Period}} % Total number of frames per loop
\providecommand\FramesCurrent{0} % Current frame number, defaults to 0

\providecommand\WormholeCircles{45} % Number of rings in the mesh
\providecommand\WormholeSectors{30} % 32 for centered line at zero rotation
\providecommand\WormholeRotations{2} % How many degrees to rotate per frame

\newcommand\WormholeRate{\eval{\WormholeRotations*(360/\WormholeSectors)/\FramesTotal}} % How many degrees to rotate per frame
\newcommand\WormholeAngle{\eval{90 + \FramesCurrent*\WormholeRate}} % Current rotation position, starting at 0

\providecommand\ClockPositionCyclesHorizontal{3}
\providecommand\ClockPositionCyclesVertical{2}
\providecommand\ClockPositionAmplitudeHorizontal{1}
\providecommand\ClockPositionAmplitudeVertical{1}

\newcommand\ClockShiftHorizontal{sin(\ClockPositionCyclesHorizontal*2*pi*\FramesCurrent/\FramesTotal)}
\newcommand\ClockShiftVertical{sin(\ClockPositionCyclesVertical*2*pi*\FramesCurrent/\FramesTotal)}

\usepackage{geometry}
\geometry{
paperwidth=16cm,
paperheight=12cm,
margin=0cm
}

\usepackage[utf8]{inputenc}
\usepackage{amsmath,amsfonts,amssymb}
\usepackage{mathrsfs}
%\usepackage{clock}
\usepackage[clock]{ifsym}
\usepackage{mathtools}
\usepackage{dcolumn}% Align table columns on decimal point
\usepackage{bm}
%\usepackage[dvipsnames]{xcolor}
\usepackage{graphicx} % Allows for eps images
\usepackage{url}
\usepackage{enumitem}
\usepackage{tensor}
\usepackage{centernot}
\usepackage{dsfont}
\usepackage[mathlines]{lineno}
\usepackage{tikz}
\usetikzlibrary{arrows,decorations.markings}
\usetikzlibrary{shapes}
\usetikzlibrary{quantikz}

\usepackage[sfdefault,medium]{inter}

\usepackage[T1]{fontenc}

%\usepackage{sansmathfonts}

\usepackage[skins]{tcolorbox}
\usepackage{tikz}
\usetikzlibrary{3d,calc,positioning}
\usepackage{pgfplots}
\usepgfplotslibrary{
    colorbrewer,
    colormaps,
    patchplots,
}
\pgfplotsset{width=16cm,compat=newest}
% Jordy blue = 138,185,241
% Pale cornflour blue = 171,205,239
\pgfplotsset{
    colormap={customcolormap}{[1cm]rgb255(0cm)=(24,115,255); rgb255(1cm)=(30,144,255); rgb255(2cm)=(108,182,255); rgb255(4cm)=(164,214,255); rgb255(6cm)=(189,224,255); rgb255(8cm)=(197,227,255); rgb255(10cm)=(210,232,255)},
    %colormap={customcolormap}{[1cm]rgb255(0cm)=(24,115,255); rgb255(1cm)=(30,144,255); rgb255(2cm)=(145,215,255); rgb255(5cm)=(195,227,255); rgb255(7cm)=(210,232,255)},
    %colormap={customcolormap}{rgb255(0cm)=(30,144,255); rgb(1cm)=(1,1,1)},
    %colormap={customcolormap}{[1cm]rgb255(0cm)=(30,144,255); rgb255(6cm)=(210,232,255); rgb255(12cm)=(210,232,255)},
    colormap={customcolormaplight}{rgb255(0cm)=(171,205,239); rgb(1cm)=(1,1,1); rgb(2cm)=(1,1,1)},
    colormap={customcolormapdark}{rgb(0cm)=(0,1,1); rgb(1cm)=(0,0,0.5); rgb(2cm)=(0,0,0.5)}
}
\newcommand\B{1}
\definecolor{meshcolor}{HTML}{FFFFFF}
\definecolor{meshcolordark}{HTML}{7FFFD4}
\definecolor{meshcolorlight}{HTML}{FFFFFF}
\definecolor{fontcolor}{HTML}{003989}
% \definecolor{fontcolor}{HTML}{224499}
\definecolor{fontcolordark}{HTML}{FFFFFF}
\definecolor{fontcolorlight}{HTML}{000000}

\usepackage[outline]{contour}
\renewcommand{\arraystretch}{20}
\contourlength{0.115pt}
\contournumber{200}

\begin{document}

\begin{tikzpicture}[remember picture, overlay]
    \begin{scope}[xshift=7.475cm, yshift=-4.15cm, scale=3.25, rotate=0, xshift=0cm, yshift=0cm]
    \begin{axis}[
    %hide axis,
    anchor=center,
    %axis equal image,
    unit vector ratio=1 1 1,
    plot box ratio=1 1 4,
    view={0}{35},
    line width=9000sp, % 2000 to 7500
    data cs=polar,
    samples=(\WormholeSectors+1),
    samples y=\WormholeCircles, %circles
    domain=\WormholeAngle:(360+\WormholeAngle),
    y domain=\B:500,
    colormap name=customcolormap,
    declare function={
        wormhole(\r)={2*\B*sqrt(\r/\B - 1)};
        % added functions to calculate cartesian coordinates from
        % polar coordinates
        pol2cartX(\angle,\radius) = \radius * cos(\angle);
        pol2cartY(\angle,\radius) = \radius * sin(\angle);
    },
]
    %\addplot3 [surf, shader=flat, draw=black, z buffer=sort, samples=50] {-wormhole(y)};
    \IfFileExists{./clock-distorted.png}{ % True

    }{ % False
    \addplot3 [surf, shader=faceted interp, faceted color = meshcolor, draw = white, z buffer=sort] {wormhole(1.5*y)};
    }
    %\addplot3 [surf, shader=interp, z buffer=sort, samples=50] {wormhole(y)};
    %\addplot3 [surf, smooth, draw=black, z buffer=sort, samples=50] {wormhole(y)};
    %\node at (60,wormhole(200)) {\includegraphics[width=12cm]{extra_fancy_clock.png}};
    \end{axis}
    \end{scope}
    \IfFileExists{./clock-distorted.png}{ % True
    \node[anchor=center,xshift=7.475cm,yshift=-3.35cm,rotate=0, opacity=1] at (0,0) {\includegraphics[height=4.15cm]{clock-distorted.png}};
    }{ % False
    \node[anchor=center,xshift=\eval{7.475 + 0.15*\ClockPositionAmplitudeHorizontal*\ClockShiftHorizontal} cm,yshift=\eval{-3.35 + 0.35*\ClockPositionAmplitudeVertical*\ClockShiftVertical} cm,rotate=0, opacity=1] at (0,0) {\includegraphics[height=4.15cm]{render-clock-fixed.png}};
    % \node[anchor=center,xshift=7.425cm,yshift=-4.7cm,rotate=0,yshift=-2cm, scale=8.35, color=fontcolor] at (0,0) {\fontfamily{ptm}\selectfont \raisebox{\iffalse\depth\fi}{Q}\tikzunderarrow[fontcolor]{\smash{hronolo}}gy};
    \node[anchor=center,xshift=7.5cm,yshift=-8.75cm,rotate=0, scale=0.5, color=fontcolor] at (0,0) {\includegraphics{logo-text.pdf}};
    \node[anchor=west,xshift=0.75cm,yshift=-8.575cm,rotate=0,yshift=-2cm, scale=1.15, color=fontcolor] at (0,0) {\contour{fontcolor}{A Python package for resolving quantum time-travel paradoxes and}};
    \node[anchor=west,xshift=0.75cm,yshift=-9.15cm,rotate=0,yshift=-2cm, scale=1.15, color=fontcolor] at (0,0) {\contour{fontcolor}{performing general quantum computation \& information processing}};
    }
\end{tikzpicture}

\end{document}
